\documentclass[]{article}

% - Style
\usepackage{base}

% - Listings
\usepackage{color}
\usepackage{listings}

\lstset{
  basicstyle=\ttfamily\footnotesize\color{black}
  , commentstyle=\color{blue}
  , keywordstyle=\color{purple}
  , stringstyle=\color{orange}
  %
  , numbers=left
  , numbersep=5pt
  , stepnumber=1
  , numberstyle=\ttfamily\small\color{black}
  %
  , keepspaces=true
  , showspaces=false
  , showstringspaces=false
  , showtabs=false
  , tabsize=2
  , breaklines=true
  %
  , frame=single
  , backgroundcolor=\color{white}
  , rulecolor=\color{black}
  , captionpos=a
}

% file or folder
\lstdefinestyle{ff}{
  basicstyle=\ttfamily\normalsize\color{orange}
}

% - Title
\title{PHYS4004 High Performance Computing - Assignment 1: OpenMP}
\author{Tom Ross - 1834 2884}
\date{}

% - Headers
\pagestyle{fancy}
\fancyhf{}
\rhead{\theauthor}
\chead{}
\lhead{\thetitle}
\rfoot{\thepage}
\cfoot{}
\lfoot{}

% - Document
\begin{document}

\tableofcontents

\newpage
\section{Overview}
\label{sec:Overview}

The codebase \lstinline[style=ff]{game-of-life}, which can be found at
\url{https://github.com/dgsaf/game-of-life}, consists of the original code
provided by Dr Pascal Elahi, with the following additions:
\begin{itemize}
\item \lstinline[style=ff]{src/02_gol_cpu_serial_fort.f90}: a serial GOL code
  which derives from \\ \lstinline[style=ff]{src/01_gol_cpu_serial_fort.f90},
  but improves loop ordering to match the column-major format of Fortran.

\item \lstinline[style=ff]{src/02_gol_cpu_openmp_loop_fort.f90}: a parallel GOL
  code which derives from \\
  \lstinline[style=ff]{src/02_gol_cpu_serial_fort.f90}, but implements OMP
  parallel do loops to yield performance benefits.

\item \lstinline[style=ff]{profiling/}: a directory which includes the profiling
  results of the \\ \lstinline[style=ff]{01_gol_cpu_serial_fort} and
  \lstinline[style=ff]{02_gol_cpu_serial_fort} versions of the GOL code, as well
  as a brief summary and comparison of the profiling results.
  The results were collected for a GOL simulation on a $1000\cross1000$ grid,
  for 100 steps with no visualisation enabled.

\item \lstinline[style=ff]{gol-job-submission.slurm}: a bash script which
  submits a \lstinline[style=ff]{sbatch} job request for a GOL simulation with a
  given set of parameters which include:
  \begin{itemize}
  \item \lstinline{version_name}

  \item \lstinline{n_omp}

  \item \lstinline{grid_height}

  \item \lstinline{grid_width}

  \item \lstinline{num_steps}

  \item \lstinline{intial_conditions_type}

  \item \lstinline{visualisation_type}

  \item \lstinline{rule_type}

  \item \lstinline{neighbour_type}

  \item \lstinline{boundary_type}
  \end{itemize}
  An output directory is created for the given set of parameters, with the
  logging output and statistics of the GOL simulation confined there.
  If the output directory already exists, the job isn't submitted to prevent
  repeating work needlessly.

\item \lstinline[style=ff]{gol-job-set-submission.sh}: a bash script which
  constructs different sets of parameters, and executes
  \lstinline[style=ff]{gol-job-submission.slurm} for each parameter set.
  The batches of jobs submitted are:
  \begin{itemize}
  \item A verification batch, which submits a job for every version of the GOL
    simulation, on a $10\cross10$ grid, for 10 steps, with ASCII visualisation.
    This is intended to allow for visual confirmation that each version produces
    uniform results.
    The logging output and statistics are compared to verify this.

  \item A scaling batch, which submits a job for every version of the GOL
    simulation, on a range of grid sizes, $2^{n}\cross2^{n}$ for
    $n = 1, \dotsc, 14$, for 100 steps, with no visualisation.
    This is intended to collect data for analysing the scaling behaviour of each
    version with increasing grid size, with the total elapsed time being
    compared.

  \item An OMP batch, which submits a job for every parallel version of the GOL
    simulation, for a range of assigned threads,
    $\mathrm{n_{omp}} = 1, \dotsc, 16$, on a $10\cross10$ grid, for 10 steps,
    with ASCII visualisation.
    This is intended to allow for visual confirmation that each parallel version
    produces uniform results, independent of the number of threads assigned to
    the program.
    The logging output and statistics are compared to verify this.

  \item An OMP batch, which submits a job for every parallel version of the GOL
    simulation, for a range of assigned threads,
    $\mathrm{n_{omp}} = 1, \dotsc, 16$, and for a range of grid sizes,
    $2^{n} \cross 2^{n}$ for $n = 1, \dotsc 14$ with no visualisation.
    This is intended to collect data for analysing the scaling behaviour of each
    parallel version, for each number of threads, with the total elapsed time
    being compared.
    Analysing this will yield some insight into the cost associated with the
    overhead of OMP threading.
    The total elapsed times are compared to investigate this.

  \end{itemize}

\item \lstinline[style=ff]{output/}: a directory which includes subdirectories
  (for each parameter set submitted), each of which include the logging output
  and statistics file; that is, \\
  \lstinline[style=ff]{output/<unique_parameter_set>/log.txt} and \\
  \lstinline[style=ff]{output/<unique_parameter_set>/stats.txt}.

\item \lstinline[style=ff]{report/}: a directory which includes this
  \lstinline[style=ff]{.tex} file and other files suitable for submission of
  this assignment.
\end{itemize}

Additionally, some minor modifications have been made to the following files:

\begin{itemize}
\item \lstinline[style=ff]{Makefile}: the make rule
  \lstinline[style=ff]{make cpu_serial_fort} has been modified to include \\
  \lstinline[style=ff]{src/02_gol_cpu_serial_fort.f90}.

\item \lstinline[style=ff]{src/common_fort.f90}: the length of the variable
  \lstinline{arg} has been increased from 32 to 2000 to allow for larger
  filenames for the variable \lstinline{statsfile}.

\item \lstinline[style=ff]{src/01_gol_cpu_serial_fort.f90}: a bug in the
  \lstinline{game_of_life_stats()} subroutine has been
\end{itemize}

\newpage
\section{Code Synopsis}
\label{sec:code-synopsis}

Here, we will provide a synopsis of the most significant changes and additions
to the codebase.
Small, but important, sections of code will be presented and discussed; the
codebase in its entirety is presented in \autoref{sec:appendix}.

\subsection{Bash Scripts}
\label{sec:bash-scripts}

\subsection{Serial Code}
\label{sec:serial-code}

\lstinputlisting[
language=Fortran
, linerange={126-127}
, firstnumber=126
]{
  ../src/01_gol_cpu_serial_fort.f90
}

\lstinputlisting[
language=Fortran
, linerange={150-152}
, firstnumber=150
]{
  ../src/01_gol_cpu_serial_fort.f90
}

\subsubsection{Original Serial Code}
\label{sec:original-serial-code}

\subsubsection{Improved Serial Code}
\label{sec:improved-serial-code}

\subsubsection{Profiling Comparison}
\label{sec:profiling-comparison}

\subsection{Parallel Loop Code}
\label{sec:parallel-loop-code}

\newpage
\section{Results}
\label{sec:results}

\subsection{Uniform Behaviour Verification}
\label{sec:unif-behav-verif}

\subsection{Scaling Behaviour}
\label{sec:scaling-behaviour}

\subsection{Thread Independence Verification}
\label{sec:thre-indep-verif}

\subsection{Thread Scaling Behaviour}
\label{sec:thread-scaling-behaviour}

\newpage
\appendix
\section{Appendix}
\label{sec:appendix}

\subsection{src/common\_fort.f90}
\label{sec:common_fort}

\lstinputlisting[
language=Fortran
]{../src/common_fort.f90}

\newpage
\subsection{src/01\_gol\_cpu\_serial\_fort.f90}
\label{sec:01_gol_cpu_serial_fort}

\lstinputlisting[
language=Fortran
]{../src/01_gol_cpu_serial_fort.f90}

\newpage
\subsection{src/02\_gol\_cpu\_serial\_fort.f90}
\label{sec:02_gol_cpu_serial_fort}

\lstinputlisting[
language=Fortran
]{../src/02_gol_cpu_serial_fort.f90}

\newpage
\subsection{src/02\_gol\_cpu\_openmp\_loop\_fort.f90}
\label{sec:02_gol_cpu_openmp_loop_fort}

\lstinputlisting[
language=Fortran
]{../src/02_gol_cpu_openmp_loop_fort.f90}

\newpage
\subsection{gol-job-submission.slurm}
\label{sec:gol-job-submission}

\lstinputlisting[
language=Bash
]{../gol-job-submission.slurm}

\newpage
\subsection{gol-job-set-submission.sh}
\label{sec:gol-job-set-submission}

\lstinputlisting[
language=Bash
]{../gol-job-set-submission.sh}

\end{document}