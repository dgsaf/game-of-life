\documentclass[]{article}

% - Style
\usepackage{base}

% \setlength{\parskip}{1 em}

% - Listings
\usepackage{color}
\usepackage{listings}

\lstset{
  basicstyle=\ttfamily\footnotesize\color{black}
  , commentstyle=\color{blue}
  , keywordstyle=\color{purple}
  , stringstyle=\color{orange}
  %
  , numbers=left
  , numbersep=5pt
  , stepnumber=1
  , numberstyle=\ttfamily\small\color{black}
  %
  , keepspaces=true
  , showspaces=false
  , showstringspaces=false
  , showtabs=false
  , tabsize=2
  , breaklines=true
  %
  , frame=single
  , backgroundcolor=\color{white}
  , rulecolor=\color{black}
  , captionpos=b
}

% file or folder
\lstdefinestyle{ff}{
  basicstyle=\ttfamily\normalsize\color{orange}
}

% - Title
\title{PHYS4004 High Performance Computing - Assignment 1: OpenMP}
\author{Tom Ross - 1834 2884}
\date{}

% - Headers
\pagestyle{fancy}
\fancyhf{}
\rhead{\theauthor}
\chead{}
\lhead{\thetitle}
\rfoot{\thepage}
\cfoot{}
\lfoot{}

% - Document
\begin{document}

\tableofcontents

\newpage
\section{Overview}
\label{sec:Overview}

The codebase \lstinline[style=ff]{game-of-life}, which can be found at
\url{https://github.com/dgsaf/game-of-life}, consists of the original code
provided by Dr Pascal Elahi, with the following additions:
\begin{itemize}
\item \lstinline[style=ff]{src/02_gol_cpu_serial_fort.f90}: a serial GOL code
  which derives from \\ \lstinline[style=ff]{src/01_gol_cpu_serial_fort.f90},
  but improves loop ordering to match the column-major format of Fortran.

\item \lstinline[style=ff]{src/02_gol_cpu_openmp_loop_fort.f90}: a parallel GOL
  code which derives from \\
  \lstinline[style=ff]{src/02_gol_cpu_serial_fort.f90}, but implements OMP
  parallel do loops to yield performance benefits.

\item \lstinline[style=ff]{profiling/}: a directory which includes the profiling
  results of the \\ \lstinline[style=ff]{01_gol_cpu_serial_fort} and
  \lstinline[style=ff]{02_gol_cpu_serial_fort} versions of the GOL code, as well
  as a brief summary and comparison of the profiling results.
  The results were collected for a GOL simulation on a $1000\cross1000$ grid,
  for 100 steps with no visualisation enabled.

\item \lstinline[style=ff]{gol-job-submission.slurm}: a SLURM script which
  submits a \lstinline[style=ff]{sbatch} job request for a GOL simulation with a
  given set of parameters which include:
  \begin{itemize}
  \item \lstinline{version_name}

  \item \lstinline{n_omp}

  \item \lstinline{grid_height}

  \item \lstinline{grid_width}

  \item \lstinline{num_steps}

  \item \lstinline{intial_conditions_type}

  \item \lstinline{visualisation_type}

  \item \lstinline{rule_type}

  \item \lstinline{neighbour_type}

  \item \lstinline{boundary_type}
  \end{itemize}
  An output directory is created for the given set of parameters, with the
  logging output and statistics of the GOL simulation confined there.
  If the output directory already exists, the job isn't submitted to prevent
  repeating work needlessly.
  Note that the \lstinline[style=ff]{sbatch} job request is invoked with
  \lstinline{--cpus-per-task=16}.

\item \lstinline[style=ff]{gol-job-set-submission.sh}: a bash script which
  constructs different sets of parameters, and executes
  \lstinline[style=ff]{gol-job-submission.slurm} for each parameter set.
  The default values for the unspecified parameters are:
  \begin{itemize}
  \item $\mathrm{n_{omp}} = 4$,
  \item $\mathrm{initial\_conditions\_type} = 0$ (random initial conditions),
  \item $\mathrm{rule\_type} = 0$ (ordinary 2D GOL rule),
  \item $\mathrm{neighbour\_type} = 0$ (Moore neighbourhood),
  \item $\mathrm{boundary\_type} = 0$ (hard boundary with default state being
    dead).
  \end{itemize}

  The batches of jobs submitted are:
  \begin{itemize}
  \item A verification batch, which submits a job for every version of the GOL
    simulation, on a $10\cross10$ grid, for 10 steps, with ASCII visualisation.
    This is intended to allow for visual confirmation that each version produces
    uniform results.
    The logging output and statistics are compared to verify this.

  \item A scaling batch, which submits a job for every version of the GOL
    simulation, on a range of grid sizes, $2^{n}\cross2^{n}$ for
    $n = 1, \dotsc, 14$, for 100 steps, with no visualisation.
    This is intended to collect data for analysing the scaling behaviour of each
    version with increasing grid size, with the total elapsed time being
    compared.

  \item An OMP batch, which submits a job for every parallel version of the GOL
    simulation, for a range of assigned threads,
    $\mathrm{n_{omp}} = 1, \dotsc, 16$, on a $10\cross10$ grid, for 10 steps,
    with ASCII visualisation.
    This is intended to allow for visual confirmation that each parallel version
    produces uniform results, independent of the number of threads assigned to
    the program.
    The logging output and statistics are compared to verify this.

  \item An OMP batch, which submits a job for every parallel version of the GOL
    simulation, for a range of assigned threads,
    $\mathrm{n_{omp}} = 1, \dotsc, 16$, and for a range of grid sizes,
    $2^{n} \cross 2^{n}$ for $n = 1, \dotsc 14$ with no visualisation.
    This is intended to collect data for analysing the scaling behaviour of each
    parallel version, for each number of threads, with the total elapsed time
    being compared.
    Analysing this will yield some insight into the cost associated with the
    overhead of OMP threading.
    The total elapsed times are compared to investigate this.

  \end{itemize}

\item \lstinline[style=ff]{output/}: a directory which includes subdirectories
  (for each parameter set submitted), each of which include the logging output
  and statistics file; that is, \\
  \lstinline[style=ff]{output/<unique_parameter_set>/log.txt} and \\
  \lstinline[style=ff]{output/<unique_parameter_set>/stats.txt}.

\item \lstinline[style=ff]{gol-data-extract.sh}: a bash script which extracts
  the final timing data for the two scaling batches of jobs:
  \begin{itemize}
  \item The scaling batch, which includes a job for every version of the GOL
    simulation, on a range of grid sizes, $2^{n} \cross 2^{n}$ for
    $n = 1, \dotsc, 14$, for 100 steps, and with no visualisation.

  \item The OMP parallel thread scaling batch, which includes a job for every
    parallel version of the GOL simulation, for a range of assigned threads,
    $\mathrm{n_{omp}} = 1, \dotsc, 16$, and for a range of grid sizes,
    $2^{n} \cross 2^{n}$, for $n = 1, \dotsc 14$, with no visualisation.
  \end{itemize}
  A data directory is created for each scaling batch of jobs, and a
  \lstinline[style=ff]{.csv} file is created for each unique job, if it doesn't
  already exist.

\item \lstinline[style=ff]{data/}: a directory which includes a subdirectory for
  each scaling batch of jobs, each of which contains a set of
  \lstinline[style=ff]{.csv} files, consisting of the grid length and timing
  data, for each of the jobs in the batch.

\item \lstinline[style=ff]{report/}: a directory which includes this
  \lstinline[style=ff]{.tex} file and other files suitable for submission of
  this assignment.
\end{itemize}

Additionally, some minor modifications have been made to the following files:

\begin{itemize}
\item \lstinline[style=ff]{Makefile}: the make rule
  \lstinline[style=ff]{make cpu_serial_fort} has been modified to include \\
  \lstinline[style=ff]{src/02_gol_cpu_serial_fort.f90}.

\item \lstinline[style=ff]{src/common_fort.f90}: the length of the variable
  \lstinline{arg} has been increased from 32 to 2000 to allow for larger
  filenames for the variable \lstinline{statsfile}.

\item \lstinline[style=ff]{src/01_gol_cpu_serial_fort.f90}: an error in
  \lstinline[language=Fortran]{subroutine game_of_life_stats()} concerning
  array indexing has been corrected.
\end{itemize}

\newpage
\section{Code Synopsis}
\label{sec:code-synopsis}

Here, we will provide a synopsis of the most significant changes and additions
to the codebase.
Small, but important, sections of code will be presented and discussed; the
codebase in its entirety is presented in \autoref{sec:appendix}.

\subsection{Common Code}
\label{sec:common-code}

The entire common code, \lstinline[style=ff]{src/common_fort.f90}, can be found
in \autoref{sec:common_fort}.

Only a small modification has been made to
\lstinline[style=ff]{src/common_fort.f90} from the original code provided by Dr
Pascal Elahi: the length of the variable \lstinline{arg} has been increased from
32 to 2000.
This is due to the slurm script \lstinline[style=ff]{gol-job-submission.slurm}
providing, as one of the command line arguments, long strings for the variable
\lstinline{statsfile}.
This is necessary for proper manipulation of the
\lstinline[style=ff]{*/stats.txt} output file, which was being truncated prior
to the change.

The modified code snippet is shown below.

\lstinputlisting[
language=Fortran
, linerange={156-157}
, firstnumber=156
, caption={Modifications to \lstinline{arg} variable within
  \lstinline{subroutine getinput()}.}
, label={lst:common_fort_snippet_1}
]{
  ../src/common_fort.f90
}

\subsection{Serial Code}
\label{sec:serial-code}

\subsubsection{Original Serial Code}
\label{sec:original-serial-code}

The entire original serial code,
\lstinline[style=ff]{src/01_gol_cpu_serial_fort.f90}, can be found in
\autoref{sec:01_gol_cpu_serial_fort}.

It contained a bug, which indexed the \lstinline{num_in_state} and
\lstinline{frac} arrays using 1-based indexing rather than 0-based indexing
(mandated by the parameters defined in
\lstinline[style=ff]{src/common_fort.f90}), resulting in incorrect statistics
being written to the statistics file.

The corrected code snippets are shown below.

\lstinputlisting[
language=Fortran
, linerange={126-130}
, firstnumber=126
, caption={Modifications to indexing of \lstinline{num_in_state} and
  \lstinline{frac} array variable within
  \lstinline{subroutine game_of_life_stats()}.}
, label={lst:01_gol_cpu_serial_fort_snippet_1}
]{
  ../src/01_gol_cpu_serial_fort.f90
}

\lstinputlisting[
language=Fortran
, linerange={153-156}
, firstnumber=153
, caption={Modifications to indexing of \lstinline{frac} array variable within
  \lstinline{subroutine game_of_life_stats()}.}
, label={lst:01_gol_cpu_serial_fort_snippet_2}
]{
  ../src/01_gol_cpu_serial_fort.f90
}

\subsubsection{Improved Serial Code}
\label{sec:improved-serial-code}

The entire improved serial code,
\lstinline[style=ff]{src/02_gol_cpu_serial_fort.f90}, can be found in
\autoref{sec:02_gol_cpu_serial_fort}.

It can be observed in \lstinline[style=ff]{src/01_gol_cpu_serial_fort.f90},
that the nested \lstinline[language=Fortran]{do} loops over the variable
\lstinline[language=Fortran]{grid(:, :)} do not respect Fortran's column major
order for multi-dimensional arrays.
Hence, it is guaranteed that cache misses are occuring as the program needs to
stride over the array in memory to access elements of
\lstinline[language=Fortran]{grid(:, :)}.
By reversing the order of these nested \lstinline[language=Fortran]{do} loops,
wherever they occur, the program will no longer suffer from these cache misses,
and will instead collect large blocks of contiguous memory, yielding performance
benefits.

The modified code snippets are shown below.

\lstinputlisting[
language=Fortran
, linerange={121-125}
, firstnumber=121
, caption={Modifications to order of nested \lstinline{do} loops within
  \lstinline{subroutine game_of_life()}.
  Mitigates problem of cache misses when accessing and updating
  \lstinline{grid(i, j)}.}
, label={lst:02_gol_cpu_serial_fort_snippet_1}
]{
  ../src/02_gol_cpu_serial_fort.f90
}

\lstinputlisting[
language=Fortran
, linerange={199-203}
, firstnumber=199
, caption={Modifications to order of nested \lstinline{do} loops within
  \lstinline{subroutine game_of_life_stats()}.
  Mitigates problem of cache misses when accessing and updating
  \lstinline{grid(i, j)}.}
, label={lst:02_gol_cpu_serial_fort_snippet_2}
]{
  ../src/02_gol_cpu_serial_fort.f90
}

\subsubsection{Profiling Comparison}
\label{sec:profiling-comparison}

The two serial codes,
\lstinline[style=ff]{src/01_gol_cpu_serial_fort.f90} and
\lstinline[style=ff]{src/02_gol_cpu_serial_fort.f90} were compiled with
profiling turned on, and performed a GOL simulation on a
$1000 \cross 1000$ grid, for 100 steps, with no visualisation.
The profiling results are located in
\lstinline[style=ff]{profiling/analysis_*.txt} and a brief summary of these
results, and the relevant source code snippets, are included in
\lstinline[style=ff]{profiling/profiling-summary.txt}.

The original serial code took \SI{4.906}{\second} to complete.
Cumulatively $\sim$\SI{45}{\percent} of the time was spent calculating the
number of living neighbours of a given cell in
\lstinline[language=Fortran]{subroutine game_of_life()},
while cumulatively $\sim$\SI{40}{\percent} of the time was spent calculating the
number of grid cells that were in a given state in
\lstinline[language=Fortran]{subroutine game_of_life_stats()}.
That is to say, $\sim$\SI{95}{\percent} of the time was spent accessing elements
of \lstinline[language=Fortran]{grid(:, :)}.
This clearly shows that the vast majority of the time spent in the GOL
simulation concerns accessing elements of
\lstinline[language=Fortran]{grid(:, :)}, and that improving the efficiency of
these access operations ensures improved code performance.

The improved serial code took \SI{2.949}{\second} to complete.
Cumulatively $\sim$\SI{65}{\percent} of the time was spent calculating the
number of living neighbours of a given cell in
\lstinline[language=Fortran]{subroutine game_of_life()},
while cumulatively only $\sim$\SI{5}{\percent} of time was spent calculating the
number of grid cells that were in a given state in
\lstinline[language=Fortran]{subroutine game_of_life_stats()}.
It is clear that reversing the order of the nested
\lstinline[language=Fortran]{do} loops has yielded a significant performance
increase, with a $\sim$\SI{40}{\percent} reduction in time taken.
Furthermore, the percentage of time spent accessing elements of
\lstinline[language=Fortran]{grid(:, :)} has been reduced from
$\sim$\SI{95}{\percent} to $\sim$\SI{70}{\percent}, indicating that the access
operations are less of a bottleneck.

Note that in \lstinline[language=Fortran]{subroutine game_of_life()},
for every update operation performed on an element of
\lstinline[language=Fortran]{grid(:, :)}, 4 to 8 access operations are
performed.

\subsection{Parallel Loop Code}
\label{sec:parallel-loop-code}

The entire parallel loop code,
\lstinline[style=ff]{src/02_gol_cpu_openmp_loop_fort.f90}, can be found in
\autoref{sec:02_gol_cpu_openmp_loop_fort}.
It is derived from \lstinline[style=ff]{src/02_gol_cpu_serial_fort.f90} but with
modifications to incorporate parallelism.

The OpenMP worksharing construct, \lstinline[language=Fortran]{parallel do}, has
been introduced to parallelise the segments of the GOL simulation code which can
be parallelised to yield performance improvements.
From the profiling analysis, discussed in \autoref{sec:profiling-comparison}, it
was observed that the majority of time is spent accessing elements the variable
\lstinline[language=Fortran]{grid(:, :)}.
Hence, the segments of code which are likely to result in the best performance
improvements when parallelised, are those that involve access operations on
\lstinline[language=Fortran]{grid(:, :)}; namely, the nested loops over
\lstinline[language=Fortran]{grid(i, j)} in
\lstinline[language=Fortran]{subroutine game_of_life()} and
\lstinline[language=Fortran]{subroutine game_of_life_stats()}.

The parallelised code snippets are shown below.

\lstinputlisting[
language=Fortran
, linerange={121-138}
, firstnumber=121
, caption={Inclusion of OMP parallel \lstinline{do} worksharing constructs
  within \lstinline{subroutine game_of_life()}.}
, label={lst:02_gol_cpu_openmp_loop_fort_snippet_1}
]{
  ../src/02_gol_cpu_openmp_loop_fort.f90
}

\lstinputlisting[
language=Fortran
, linerange={213-239}
, firstnumber=213
, caption={Inclusion of OMP parallel \lstinline{do} worksharing constructs
  within \lstinline{subroutine game_of_life_stats()}.}
, label={lst:02_gol_cpu_openmp_loop_fort_snippet_2}
]{
  ../src/02_gol_cpu_openmp_loop_fort.f90
}

Static schedulers were chosen for the \lstinline[language=Fortran]{parallel do}
loops, as the amount of work per iteration (of either loop) is expected to be
approximately constant.
Thus, any benefit from introducing a dynamic or guided scheduler would be
minimal and outweighed by the cost of introducing scheduler overhead.
A reduction clause is attached to the variable
\lstinline[language=Fortran]{num_in_state(:)}, since it is effectively an
indexed counter which can aggregate the results of each thread together without
concern.

It was determined in some small scale tests that including the clause
\lstinline{collapse(2)}, which would collapse the two perfectly-nested loops
into one, generally did not produce a significant performance benefit and in
some cases worsened performance.
However this was only tested rudimentarily and may be more signficant for
increasingly large grid sizes.

\subsection{Bash Scripts}
\label{sec:bash-scripts}

\subsubsection{GOL Job Submission}
\label{sec:gol-job}

The entire GOL job submission SLURM script,
\lstinline[style=ff]{gol-job-submission.slurm}, can be found in
\autoref{sec:gol-job-submission}.

\subsubsection{GOL Job Set}
\label{sec:gol-job-set}

The entire GOL job set submission script,
\lstinline[style=ff]{gol-job-set-submission.sh}, can be found in
\autoref{sec:gol-job-set-submission}.

\subsubsection{GOL Data Extract}
\label{sec:gol-data}

The entire GOL data extract script
\lstinline[style=ff]{gol-data-extract.sh}, can be found in
\autoref{sec:gol-data-extract}.

\newpage
\section{Results}
\label{sec:results}

\subsection{Uniform Behaviour Verification}
\label{sec:unif-behav-verif}

All versions of the GOL simulation were executed on a $10 \cross 10$ grid, for
10 steps, with ASCII visualisation on, and with $\mathrm{n_{omp}} = 4$ for the
parallel versions of the code.
The behaviour of the different versions can be compared exactly by examining the
ASCII visualisation of the grids to show that they are identical, and/or by
comparing the statistics of the grids produced.

To compare the ASCII visualisations for each version, using the \lstinline{diff}
command to compare the \lstinline[style=ff]{output/*ngrid-10x10*/log.txt} output
for each version from that of \\
\lstinline[style=ff]{src/01_gol_cpu_serial_fort.f90} should indicate any
differences in behaviour.
If each version produces identical grids, the only differences should be timing
results.

\begin{lstlisting}[language=Bash]
  original="output/GOL-01_gol_cpu_serial_fort.nomp-4.ngrid-10x10.nsteps-10.\
  ic_type-0.vis_type-0.rule_type-0.nghbr_type-0.bndry_type-0/log.txt"

  for log in output/*ngrid-10x10*/log.txt; do
    diff ${original} ${log}
  done
\end{lstlisting}

Similarly, to compare the statistics, we again use the \lstinline{diff} command
to compare the \lstinline[style=ff]{output/*ngrid-10x10*/stats.txt} output
for each version from that of
\lstinline[style=ff]{src/01_gol_cpu_serial_fort.f90}.

\begin{lstlisting}[language=Bash]
  original="output/GOL-01_gol_cpu_serial_fort.nomp-4.ngrid-10x10.nsteps-10.\
  ic_type-0.vis_type-0.rule_type-0.nghbr_type-0.bndry_type-0/stats.txt"

  for stats in output/*ngrid-10x10*/stats.txt; do
    diff ${original} ${stats}
  done
\end{lstlisting}

Using both methods on the data collected in \lstinline[style=ff]{output/}, we
have verified that the different versions produce identical grids and
statistics.

\subsection{Scaling Behaviour}
\label{sec:scaling-behaviour}

\subsection{Thread Independence Verification}
\label{sec:thre-indep-verif}

All parallel versions of the GOL simulation were executed on a $10 \cross 10$
grid, for 10 steps, with ASCII visualisation on, and for a range of assigned
threads $\mathrm{n_{omp}} = 1, \dotsc, 16$.
The behaviour of the different versions can be compared exactly by examining the
ASCII visualisation of the grids to show that they are identical, and/or by
comparing the statistics of the grids produced.

The ASCII visualisations and statistics can be compared by using the
\lstinline[language=Bash]{diff} command in an almost identical way as in
\autoref{sec:unif-behav-verif}.

\begin{lstlisting}[language=Bash]
  original="output/GOL-02_gol_cpu_openmp_loop_fort.nomp-1.ngrid-10x10.nsteps-10.\
  ic_type-0.vis_type-0.rule_type-0.nghbr_type-0.bndry_type-0/log.txt"

  for log in output/*openmp*ngrid-10x10*/log.txt; do
    diff ${original} ${log}
  done
\end{lstlisting}

\begin{lstlisting}[language=Bash]
  original="output/GOL-02_gol_cpu_openmp_loop_fort.nomp-1.ngrid-10x10.nsteps-10.\
  ic_type-0.vis_type-0.rule_type-0.nghbr_type-0.bndry_type-0/stats.txt"

  for stats in output/*openmp*ngrid-10x10*/stats.txt; do
    diff ${original} ${stats}
  done
\end{lstlisting}

Using both methods on the data collected in \lstinline[style=ff]{output/}, we
have verified that the parallel versions produce identical grids and statistics,
independently of the number of OMP threads assigned.


\subsection{Thread Scaling Behaviour}
\label{sec:thread-scaling-behaviour}

\newpage
\appendix
\section{Appendix}
\label{sec:appendix}

\subsection{src/common\_fort.f90}
\label{sec:common_fort}

\lstinputlisting[
language=Fortran
]{../src/common_fort.f90}

\newpage
\subsection{src/01\_gol\_cpu\_serial\_fort.f90}
\label{sec:01_gol_cpu_serial_fort}

\lstinputlisting[
language=Fortran
]{../src/01_gol_cpu_serial_fort.f90}

\newpage
\subsection{src/02\_gol\_cpu\_serial\_fort.f90}
\label{sec:02_gol_cpu_serial_fort}

\lstinputlisting[
language=Fortran
]{../src/02_gol_cpu_serial_fort.f90}

\newpage
\subsection{src/02\_gol\_cpu\_openmp\_loop\_fort.f90}
\label{sec:02_gol_cpu_openmp_loop_fort}

\lstinputlisting[
language=Fortran
]{../src/02_gol_cpu_openmp_loop_fort.f90}

\newpage
\subsection{gol-job-submission.slurm}
\label{sec:gol-job-submission}

\lstinputlisting[
language=Bash
]{../gol-job-submission.slurm}

\newpage
\subsection{gol-job-set-submission.sh}
\label{sec:gol-job-set-submission}

\lstinputlisting[
language=Bash
]{../gol-job-set-submission.sh}

\newpage
\subsection{gol-data-extract.sh}
\label{sec:gol-data-extract}

\lstinputlisting[
language=Bash
]{../gol-data-extract.sh}

\end{document}