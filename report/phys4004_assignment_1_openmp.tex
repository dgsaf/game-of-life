\documentclass[draft]{article}

% - Style
\usepackage{base}

% - Listings
\usepackage{color}
\usepackage{listings}

\lstset{
  basicstyle=\ttfamily\normalsize\color{orange}
  , commentstyle=\color{blue}
  , keywordstyle=\color{red}
  , stringstyle=\color{purple}
  %
  , numbers=left
  , numbersep=5pt
  , numberstyle=\ttfamily\tiny\color{pink}
  , stepnumber=1
  %
  , showspaces=false
  , showstringspaces=false
  , showtabs=false,
  , tabsize=2
  %
  , frame=single
  , keepspaces=true
  , backgroundcolor=\color{white}
  , rulecolor=\color{black}
  , captionpos=b
}

% - Title
\title{PHYS4004 High Performance Computing - Assignment 1: OpenMP}
\author{Tom Ross - 1834 2884}
\date{}

% - Headers
\pagestyle{fancy}
\fancyhf{}
\rhead{\theauthor}
\chead{}
\lhead{\thetitle}
\rfoot{\thepage}
\cfoot{}
\lfoot{}

% - Document
\begin{document}

\section{Overview}
\label{sec:Overview}

The following items are included:
\begin{itemize}
\item The codebase \lstinline{game-of-life} consists of the original code
  provided by Dr Pascal Elahi, with the following additions:
  \begin{itemize}
  \item \lstinline{src/02_gol_cpu_serial_fort.f90}: a serial GOL code which
    derives from \\ \lstinline{src/01_gol_cpu_serial_fort.f90}, but improves
    loop ordering to match the column-major format of Fortran.

  \item \lstinline{src/02_gol_cpu_openmp_loop_fort.f90}: a parallel GOL code which
    derives from \\ \lstinline{src/02_gol_cpu_serial_fort.f90}, but implements
    OMP parallel do loops to yield performance benefits.

  \item \lstinline{profiling/}: a directory which includes the profiling results
    of the \\ \lstinline{01_gol_cpu_serial_fort} and
    \lstinline{02_gol_cpu_serial_fort} versions of the GOL code, as well as a
    brief summary and comparison of the profiling results.
    The results were collected for a GOL simulation on a $1000\cross1000$ grid,
    for 100 steps with no visualisation enabled.

  \item \lstinline{gol-job-submission.slurm}: a bash script which submits a
    \lstinline{sbatch} job request for a GOL simulation with a given set of
    parameters which include:
    \begin{itemize}
    \item \lstinline{version_name}

    \item \lstinline{n_omp}

    \item \lstinline{grid_height}

    \item \lstinline{grid_width}

    \item \lstinline{num_steps}

    \item \lstinline{intial_conditions_type}

    \item \lstinline{visualisation_type}

    \item \lstinline{rule_type}

    \item \lstinline{neighbour_type}

    \item \lstinline{boundary_type}
    \end{itemize}
    An output directory is created for the given set of parameters, with the
    logging output and statistics of the GOL simulation confined there.
    If the output directory already exists, the job isn't submitted to prevent
    repeating work needlessly.

  \item \lstinline{gol-job-set-submission.sh}: a bash script which constructs
    different sets of parameters, and executes
    \lstinline{gol-job-submission.slurm} for each parameter set.
    Three batches of jobs are submitted:
    \begin{itemize}
    \item A verification batch, which submits a job for every version of the GOL
      simulation, on a $10\cross10$ grid, for 10 steps, with ASCII
      visualisation.
      This is intended to allow for visual confirmation that each version
      produces uniform results.
      The logging output and statistics are compared to verify this.

    \item A scaling batch, which submits a job for every version of the GOL
      simulation, on a range of grid sizes, $2^{n}\cross2^{n}$ for
      $n = 1, \dotsc, 14$, for 100 steps, with no visualisation.
      This is intended to collect data for analysing the scaling behaviour of
      each version with increasing grid size, with the total elapsed time being
      compared.

    \item An OMP batch, which submits a job for every parallel version of the
      GOL simulation, for a range of assigned threads,
      $\mathrm{n_{omp}} = 1, \dotsc, 16$, on a $10\cross10$ grid, for 10 steps,
      with ASCII visualisation.
      This is intended to allow for visual confirmation that each parallel
      version produces uniform results, independent of the number of threads
      assigned to the program.
      The logging output and statistics are compared to verify this.
    \end{itemize}

  \item \lstinline{output/}: a directory which includes subdirectories (for each
    parameter set submitted), each of which include the logging output and
    statistics file; that is, \\
    \lstinline{output/<unique_parameter_set>/log.txt} and \\
    \lstinline{output/<unique_parameter_set>/stats.txt}.

  \item \lstinline{report/}: a directory which includes this \lstinline{.tex}
    file and other files suitable for submission of this assignment.
  \end{itemize}

  Additionally, some minor modifications have been made to the following files:

  \begin{itemize}
  \item \lstinline{Makefile}: the make rule \lstinline{make cpu_serial_fort} has
    been modified to include \\ \lstinline{src/02_gol_cpu_serial_fort.f90}.

  \item \lstinline{src/common_fort.f90}: the length of the variable
    \lstinline[basicstyle=\ttfamily\color{blue}]{arg} has been increased from 32
    to 2000 to allow for larger filenames for the variable
    \lstinline[basicstyle=\ttfamily\color{blue}]{statsfile}.
  \end{itemize}

\item Instances of codebase are provided in both zip and tar.gz formats, and
  can also be found at \url{https://github.com/dgsaf/game-of-life}.
\end{itemize}

\section{Serial Code}
\label{sec:serial-code}

\subsection{Original Serial Code}
\label{sec:original-serial-code}

\subsection{Improved Serial Code}
\label{sec:improved-serial-code}

\subsection{Profiling Comparison}
\label{sec:profiling-comparison}

\section{Loop Parallel Code}
\label{sec:loop-parallel-code}

\section{Results}
\label{sec:results}

\subsection{Uniform Behaviour Verification}
\label{sec:unif-behav-verif}

\subsection{Scaling Behaviour}
\label{sec:scaling-behaviour}

\subsection{Thread Independence Verification}
\label{sec:thre-indep-verif}

\end{document}